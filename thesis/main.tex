%\documentclass[twoside,english]{/usr/local/texlive/2012/texmf-dist/tex/latex/uiofysmaster/uiofysmaster}
\documentclass[twoside,english]{uiofysmaster}
\usepackage{epstopdf}
\usepackage{pstool}
\graphicspath{{/Users/anderhaf/Library/texmf/tex/latex/uiofysmaster//}}
\bibliography{references}

\author{Anders Hafreager}
\title{\uppercase{DSMC}}
\date{October 2013}

\begin{document}

\maketitle

\begin{abstract}
This is an abstract text.
\end{abstract}

\begin{dedication}
  To someone
  \\\vspace{12pt}
  This is a dedication to my cat.
\end{dedication}

\begin{acknowledgements}
  I acknowledge my acknowledgements.
\end{acknowledgements}

\tableofcontents
\clearpage
\listoffigures
\clearpage
\listoftables

\chapter{The beginning is here}
\section{Interaction with the surface}
In nanoporous media, the surface area is large compared to the volume as stated by the square-cube law. In addition, dilute gases have a rather large mean free path, hence the interactions with the surface is very important in such systems. We therefore need a good model for such interactions. The simplest mo
\subsection{Specular reflection}
\subsection{Diffuse reflection}
\subsection{Maxwell kernel}
\subsection{Cercignani-Lampis model}
In order to have a more realistic kernel

\end{document}