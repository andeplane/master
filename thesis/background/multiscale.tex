\section{Multi scale physics}

\subsection{Our world}
Throughout the history, from the ancient greeks up until today, humans have tried to understand the rules of our universe and how they affect what we see and experience every day. Today, we know that everything is controlled by the rules of quantum mechanics, where some of its effects are visible for the naked eye. An example is seen every summer during hot days while driving a car on a straight road; mirage. The warm road reflects light as if it's covered in water, so you can see the sky and oncoming cars. This phenomena is explained by quantum electrodynamics and the fact that there is a temperature gradient in the air pointing from the asphalt and upwards. Light moves slower through dense air, and the air is less dense at high temperatures, so the light wants to travel closer to the asphalt. This means that the actual path the light takes is not a straight line, but is bent, see figure \ref{TODO}. The light is taking a shortcut. Mirage can be explained, or should I say modelled, by simpler ideas, but the actual reason arises from quantum mechanics. 

\subsection{Simulation of the universe}
Say we want to use our computers and simulate a universe very similar to our own. We will use all our knowledge about physics, so the fundamental theory we plug in is quantum mechanics. We need to account for relativistic effects, so we have to unify quantum theory with general theory of relativity first. A universe, even a tiny one, contains a lot of information, so the amount of required computer memory is enormous. In order to solve the fundamental equations of unified relativistic quantum mechanics with todays techniques, we will probably need a very small timestep. Larger time scales, like the ones needed to compute cosmological properties of the universe, are therefore out of reach even with decades of Moore's law. 

\subsection{An ideal world}

\subsection{The weakest link}

