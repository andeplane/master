%&latex
\documentclass{article}

\usepackage[T1]{fontenc}
\usepackage{graphicx}
\usepackage{amsmath}
\usepackage{amssymb}
\usepackage{amsbsy}
\usepackage{amsfonts}
\usepackage{color}
\usepackage{bbm}
\usepackage{epstopdf}

\newcommand{\eqsN}{\begin{equation*}}
\newcommand{\eqeN}{\end{equation*}}
\newcommand{\eqs}{\begin{equation}}
\newcommand{\eqe}{\end{equation}}
\newcommand{\diff}[2]{\frac{\mathrm{d}#1}{\mathrm{d}#2}}
\newcommand{\esum}[3]{\sum_{#1=#2}^#3}
\newcommand{\nsum}{\esum{n}{0}{\infty}}
\newcommand{\prt}[1]{\frac{\partial}{\partial#1}}
\newcommand{\prtt}[1]{\frac{\partial^2}{\partial#1^2}}
\newcommand{\f}[2]{\frac{#1}{#2}}
\newcommand{\dd}[1]{\ \text{d}#1}
\begin{document}
\author{Anders Hafreager}
\title{DSMC}
\maketitle
Vi starter med å lage $n$ partikler med hastigheter fordelt med Maxwell Speed Distribution, altså normalfordeling i hver dimensjon med $\sigma = \sqrt{kT/m}$. Ved en gitt massetetthet $\rho$ (1.78 kg/m$^3$ for argon) kan vi regne hvor mange ekte atomer hver partikkel $i$ tilsvarer ved $eff = \rho/m*L^3/n$. 
Vi antar at vi deler opp volumet i $k$ celler og velger et tidssteg $dt$ slik at vi har god oppløsning i hver celle.
\end{document}